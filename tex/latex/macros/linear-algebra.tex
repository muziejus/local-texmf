\newcommand{\mymat}[1]{%
  \begin{bmatrix*}[r]
    #1
  \end{bmatrix*}
}% end \mymat

\NewDocumentCommand{\colvecn}{ O{x} O{3} }{%
  \begin{bmatrix}
   \iteratevecn{#1}{#2}
  \end{bmatrix}
}% end \colvecn

\newcounter{vecncounter}
\newbool{vecnlength}
\newcommand{\iteratevecn}[2]{
  \defcounter{vecncounter}{0}
  \unlessboolexpr{bool {vecnlength}}{
    \stepcounter{vecncounter}
    \ifnumcomp{\thevecncounter}{>}{#2}
      {\setbool{vecnlength}{true}}
      {
        \ifnumcomp{\thevecncounter}{=}{#2}
          {{#1}_{\thevecncounter}}
          {{#1}_{\thevecncounter} \\ }
      }
  }% end \unlessboolexpr
}% end \iteratevecn


\newcommand{\colvecton}[1][x]{%
  \begin{bmatrix}[c]
    {#1}_1 \\ \vdots \\ {#1}_n
  \end{bmatrix}
}% end \colvecton


\DeclareMathOperator{\rref}{rref}
\DeclareMathOperator{\proj}{proj}
\DeclareMathOperator{\reff}{ref}
\DeclareMathOperator{\image}{image}
\DeclareMathOperator{\kerr}{ker}
\DeclareMathOperator{\rank}{rank}
\DeclareMathOperator{\trace}{tr}

% For augmented matrices
\makeatletter
\renewcommand*\env@matrix[1][*\c@MaxMatrixCols c]{%
  \hskip -\arraycolsep
  \let\@ifnextchar\new@ifnextchar
  \array{#1}}
\makeatother
